\documentclass{beamer}

\author[D. Abercrombie]{
  Daniel Abercrombie
}

\title{\bf \sffamily Attempting V-Tagging Scale Factors for CA15 Jets}
\date{\today}

\usecolortheme{dove}

\usepackage[absolute,overlay]{textpos}
\usefonttheme{serif}
\usepackage{appendixnumberbeamer}
\usepackage{isotope}
\usepackage{hyperref}
\usepackage[english]{babel}
\usepackage{amsmath}
\setbeamerfont{frametitle}{size=\Large,series=\bf\sffamily}
\setbeamertemplate{frametitle}[default][center]
\usepackage{siunitx}
\usepackage{tabularx}
\usepackage{makecell}

\setbeamertemplate{navigation symbols}{}
\usepackage{graphicx}
\usepackage{color}
\setbeamertemplate{footline}[text line]{\parbox{1.083\linewidth}{\footnotesize \hfill \insertshortauthor \hfill \insertpagenumber /\inserttotalframenumber}}
\setbeamertemplate{headline}[text line]{\parbox{1.083\linewidth}{\footnotesize \hspace{-0.083\linewidth} \textcolor{blue}{\sffamily \insertsection \hfill \insertsubsection}}}

\IfFileExists{/Users/dabercro/GradSchool/Presentations/MIT-logo.pdf}
             {\logo{\includegraphics[height=0.5cm]{/Users/dabercro/GradSchool/Presentations/MIT-logo.pdf}}}
             {\logo{\includegraphics[height=0.5cm]{/home/dabercro/MIT-logo.pdf}}}

\usepackage{changepage}

\newcommand{\beginbackup}{
  \newcounter{framenumbervorappendix}
  \setcounter{framenumbervorappendix}{\value{framenumber}}
}
\newcommand{\backupend}{
  \addtocounter{framenumbervorappendix}{-\value{framenumber}}
  \addtocounter{framenumber}{\value{framenumbervorappendix}}
}

\graphicspath{{figs/}}

\begin{document}

\begin{frame}[nonumbering]
  \titlepage
\end{frame}

\begin{frame}
  \frametitle{Introduction}

  We attempt a simple cut and count measurement of the V-tagging scale factor for CA15 jets.

  This method requires that the mass and substructure distributions of the contribution this factor is applied to
  matches the ``signal'' distributions in these slides.
  If that is not the case, we would not have what could be considered a ``pure'' sample resembling that contribution.

  For more details on this reasoning, see slide 10 of
  \href{http://t3serv001.mit.edu/~dabercro/docs/WTagStudy/dabercro_WTagStudy_160727.pdf}
       {\textcolor{blue}{these slides}}.

\end{frame}

\begin{frame}
  \frametitle{Samples Used (Sort of)}

  Benedikt can answer questions about the details, but the main backgrounds used are:

  \begin{itemize}
  \item \ttbar + jets
  \item Single top
  \item W/Z + jets
  \item Diboson
  \item QCD
  \end{itemize}

\end{frame}

\begin{frame}
  \frametitle{Base Selection}

  \begin{itemize}
  \item Exactly one fat jet with $p_T > \SI{200}{GeV}$ and $50 < m_{SD} < \SI{250}{GeV}$
  \item $\Delta\phi\mathrm{(fat jet, recoil)} > 0.4$
  \item Two loose-tagged b jets, with at least one medium, not inside the fat jet
  \item Fewer than 4 isolated narrow (AK4) jets
  \item One tight muon and no electrons, taus, or photons
  \item Passes MET filter
  \end{itemize}

\end{frame}

\begin{frame}
  \frametitle{Mass at Base Selection}

  In the base selection, we get a large amount of merged tops resulting in a top mass peak.
  We need to add some purity cut to this.
  We will consider:
  \begin{itemize}
  \item Sub-jet b-tag veto
  \item $\Delta R$ cut between fat jet and closest b-tagged jet
  \end{itemize}

  \centering
  \includegraphics[width=0.6\linewidth]{170207_VTagCA15/cat_base_fj1MSD_corr.pdf}

\end{frame}

\begin{frame}
  \frametitle{Sub-jet b-tag Veto}

  Require that the maximum sub-jet csv$ < 0.5$.

  \vspace{6pt}

  \begin{columns}
    \begin{column}{0.5\linewidth}
      \includegraphics[width=\linewidth]{170207_VTagCA15/cat_fullcutz_fj1MaxCSV.pdf}
    \end{column}
    \begin{column}{0.5\linewidth}
      \includegraphics[width=\linewidth]{170207_VTagCA15/cat_fullcutz_fj1MSD_corr.pdf}
    \end{column}
  \end{columns}

  \vspace{6pt}

  Purity of W-matched MC: $5697.0/\SI{20703.4}{events} = 27.5\%$

\end{frame}

\begin{frame}
  \frametitle{$\Delta R$ Between Fat Jet and b-tagged Jet}

  Require that $\DeltaR > 1.5$.

  \vspace{6pt}

  \begin{columns}
    \begin{column}{0.5\linewidth}
      \includegraphics[width=\linewidth]{170207_VTagCA15/cat_base+farB_dRfj1Isob.pdf}
    \end{column}
    \begin{column}{0.5\linewidth}
      \includegraphics[width=\linewidth]{170207_VTagCA15/cat_base+farB_fj1MSD_corr.pdf}
    \end{column}
  \end{columns}

  \vspace{6pt}

  Purity of W-matched MC: $5516.1/\SI{20106.6}{events} = 27.4\%$

\end{frame}

\begin{frame}
  \frametitle{Cuts are Similar}

  Below shows how each distribution is affected by the cut on the other variable.

  \vspace{6pt}

  \begin{columns}
    \begin{column}{0.5\linewidth}
      \textcolor{blue}{$\Delta R$ with a sub-b veto} \\
      \includegraphics[width=\linewidth]{170207_VTagCA15/cat_fullcutz+farB_dRfj1Isob.pdf}
    \end{column}
    \begin{column}{0.5\linewidth}
      \textcolor{blue}{Max sub-csv with a $\Delta R$ cut} \\
      \includegraphics[width=\linewidth]{170207_VTagCA15/cat_fullcutz+farB_fj1MaxCSV.pdf}
    \end{column}
  \end{columns}

  \vspace{6pt}

  We will use the subjet b-tag veto for the measurement and the $\Delta R$ as a systematic,
  along with the base selection.

\end{frame}

\begin{frame}
  \frametitle{Tagging Variables Before Tagging}

  The Mass and DDT distributions before tagging are shown below

  \vspace{12pt}

  \begin{columns}
    \begin{column}{0.5\linewidth}
      \includegraphics[width=\linewidth]{170207_VTagCA15/cat_fullcutz_fj1MSD_corr.pdf}
    \end{column}
    \begin{column}{0.5\linewidth}
      \includegraphics[width=\linewidth]{170207_VTagCA15/cat_fullcutz_N2DDT.pdf}
    \end{column}
  \end{columns}

\end{frame}

\begin{frame}
  \frametitle{Tagging Variables After Tagging}

  The Mass and DDT variables shape each other when using the tagging cuts

  \vspace{12pt}

  \begin{columns}
    \begin{column}{0.5\linewidth}
      \includegraphics[width=\linewidth]{170207_VTagCA15/cat_fullcutz+rightMass+n2ddt_fj1MSD_corr.pdf}
    \end{column}
    \begin{column}{0.5\linewidth}
      \includegraphics[width=\linewidth]{170207_VTagCA15/cat_fullcutz+rightMass+n2ddt_N2DDT.pdf}
    \end{column}
  \end{columns}

\end{frame}

\begin{frame}
  \frametitle{Scale Factors}

  \begin{columns}
    \begin{column}{0.5\linewidth}
      \centering
      \includegraphics[width=0.8\linewidth]{170207_VTagCA15_background/cat_fullcutz_fj1MSD_corr.pdf}
    \end{column}
    \begin{column}{0.5\linewidth}
      \centering
      \includegraphics[width=0.8\linewidth]{170207_VTagCA15_background/cat_fullcutz_N2DDT.pdf}
    \end{column}
  \end{columns}

  \textcolor{red}{V-tagging cuts: $65 < m_{SD} < \SI{105}{GeV}$ and $N2DDT < 0.0$}

  \begin{adjustwidth}{-2.5em}{-2.5em}
    \centering

    {\scriptsize
      \begin{tabular}{| c | c | c | c | c |}
        \hline
        & No Cut & SD Mass Cut & N2DDT & Full Cut \\
        \hline
        \makecell{Background \\ Subtracted \\ Data} & 1683.17 $\pm$ 48.84 & 671.28 $\pm$ 30.50 & 548.56 $\pm$ 28.46 & 332.68 $\pm$ 20.12 \\
        \makecell{Signal-\\ matched MC} & 1666.66 $\pm$ 22.84 & 655.41 $\pm$ 16.55 & 517.59 $\pm$ 14.82 & 307.63 $\pm$ 12.79 \\
        \hline
        \makecell{Normalized \\ Ratio} & 1.00 $\pm$ 0.03 & 1.01 $\pm$ 0.05 & 1.05 $\pm$ 0.06 & \fcolorbox{red}{yellow}{1.07 $\pm$ 0.08} \\
        \hline
      \end{tabular}
    }
  \end{adjustwidth}

\end{frame}

\begin{frame}
  \frametitle{Background Scaling}

  Scaling the background up by a factor of two made a larger difference than scaling down
  by a factor of two.

  \begin{columns}
    \begin{column}{0.5\linewidth}
      \centering
      \includegraphics[width=0.8\linewidth]{170207_VTagCA15_background/cat_fullcutz_fj1MSD_corr.pdf}
    \end{column}
    \begin{column}{0.5\linewidth}
      \centering
      \includegraphics[width=0.8\linewidth]{170207_VTagCA15_up/cat_fullcutz_fj1MSD_corr.pdf}
    \end{column}
  \end{columns}

  \textcolor{red}{V-tagging cuts: $65 < m_{SD} < \SI{105}{GeV}$ and $N2DDT < 0.0$}

  \begin{adjustwidth}{-2.5em}{-2.5em}
    \centering

    {\scriptsize
      \begin{tabular}{| c | c | c | c | c |}
        \hline
        & No Cut & SD Mass Cut & N2DDT & Full Cut \\
        \hline
        \makecell{Background \\ Subtracted \\ Data} & 1511.34 $\pm$ 63.04 & 608.55 $\pm$ 38.97 & 493.12 $\pm$ 37.78 & 312.36 $\pm$ 23.66 \\
        \makecell{Signal-\\ matched MC} & 1666.66 $\pm$ 22.84 & 655.41 $\pm$ 16.55 & 517.59 $\pm$ 14.82 & 307.63 $\pm$ 12.79 \\
        \hline
        \makecell{Normalized \\ Ratio} & 1.00 $\pm$ 0.04 & 1.02 $\pm$ 0.07 & 1.05 $\pm$ 0.09 & \fcolorbox{red}{yellow}{1.12 $\pm$ 0.10} \\
        \hline
      \end{tabular}
    }
  \end{adjustwidth}

\end{frame}

\begin{frame}
  \frametitle{Background Selection}

  By saying that only W-matched and Single top are signal-like, we get this alternate scale factor

  \begin{columns}
    \begin{column}{0.5\linewidth}
      \centering
      \includegraphics[width=0.8\linewidth]{170207_VTagCA15_background/cat_fullcutz_fj1MSD_corr.pdf}
    \end{column}
    \begin{column}{0.5\linewidth}
      \centering
      \includegraphics[width=0.8\linewidth]{170207_VTagCA15_more/cat_fullcutz_fj1MSD_corr.pdf}
    \end{column}
  \end{columns}

  \textcolor{red}{V-tagging cuts: $65 < m_{SD} < \SI{105}{GeV}$ and $N2DDT < 0.0$}

  \begin{adjustwidth}{-2.5em}{-2.5em}
    \centering

    {\scriptsize
      \begin{tabular}{| c | c | c | c | c |}
        \hline
        & No Cut & SD Mass Cut & N2DDT & Full Cut \\
        \hline
        \makecell{Background \\ Subtracted \\ Data} & 744.68 $\pm$ 50.95 & 431.91 $\pm$ 31.81 & 314.05 $\pm$ 29.84 & 256.95 $\pm$ 21.38 \\
        \makecell{Signal-\\ matched MC} & 737.37 $\pm$ 17.74 & 418.39 $\pm$ 13.93 & 285.38 $\pm$ 11.86 & 232.65 $\pm$ 10.60 \\
        \hline
        \makecell{Normalized \\ Ratio} & 1.00 $\pm$ 0.07 & 1.02 $\pm$ 0.08 & 1.09 $\pm$ 0.11 & \fcolorbox{red}{yellow}{1.09 $\pm$ 0.10} \\
        \hline
      \end{tabular}
    }
  \end{adjustwidth}

\end{frame}

\begin{frame}
  \frametitle{Base Selection and Top $p_T$ Reweighting}

  Using the base selection \\ (no b-tag veto, an aggressive systematic):

  \begin{adjustwidth}{-2.5em}{-2.5em}
    \centering

    {\scriptsize
      \begin{tabular}{| c | c | c | c | c |}
        \hline
        & No Cut & SD Mass Cut & N2DDT & Full Cut \\
        \hline
        \makecell{Background \\ Subtracted \\ Data} & 3449.58 $\pm$ 67.74 & 1145.44 $\pm$ 38.15 & 990.28 $\pm$ 37.70 & 530.19 $\pm$ 24.94 \\
        \makecell{Signal-\\ matched MC} & 3293.68 $\pm$ 31.08 & 1070.16 $\pm$ 20.21 & 937.79 $\pm$ 18.55 & 495.15 $\pm$ 15.11 \\
        \hline
        \makecell{Normalized \\ Ratio} & 1.00 $\pm$ 0.02 & 1.02 $\pm$ 0.04 & 1.01 $\pm$ 0.04 & \fcolorbox{red}{yellow}{1.02 $\pm$ 0.06} \\
        \hline
      \end{tabular}
    }
  \end{adjustwidth}

  Turning off top $p_T$ reweighting:

  \begin{adjustwidth}{-2.5em}{-2.5em}
    \centering

    {\scriptsize
      \begin{tabular}{| c | c | c | c | c |}
        \hline
        & No Cut & SD Mass Cut & N2DDT & Full Cut \\
        \hline
        \makecell{Background \\ Subtracted \\ Data} & 1703.56 $\pm$ 47.61 & 678.72 $\pm$ 29.77 & 555.14 $\pm$ 27.64 & 335.09 $\pm$ 19.83 \\
        \makecell{Signal-\\ matched MC} & 1913.87 $\pm$ 24.83 & 753.04 $\pm$ 17.63 & 592.50 $\pm$ 15.75 & 354.99 $\pm$ 13.47 \\
        \hline
        \makecell{Normalized \\ Ratio} & 1.00 $\pm$ 0.03 & 1.01 $\pm$ 0.05 & 1.05 $\pm$ 0.06 & \fcolorbox{red}{yellow}{1.06 $\pm$ 0.07} \\
        \hline
      \end{tabular}
    }
  \end{adjustwidth}

\end{frame}

\begin{frame}
  \frametitle{Systematic Uncertainties}
  We assume systematic uncertainties are symmetric for now
  \begin{center}
  \begin{tabular}{l|c}
    Source & Uncertainty \\
    \hline
    Background Subtraction & 5\% \\
    Which background is subtracted & 2\% \\
    B-veto selection & 5\% (rather aggressive) \\
    Top $p_T$ reweighting & 1\% \\
  \end{tabular}
  \end{center}
  Adding in Quadrature: \boxed{$\pm 7.4\%$}
  \vspace{12pt}
\end{frame}

\begin{frame}
  \frametitle{Conclusions}

  \begin{itemize}
  \item No obvious way to make the selection cleaner to model the actual signal events,
    but useful for signal-like backgrounds with same distributions
  \item With current limited systematics, we get:
    \begin{center}
      \boxed{$1.07 \pm 0.08 \mathrm{(stat.)} \pm 0.07 \mathrm{(sys.)}$}
    \end{center}
  \end{itemize}

\end{frame}

\beginbackup

\begin{frame}
  \frametitle{Backup Slides}
\end{frame}

\begin{frame}
   \frametitle{\small 170207\_VTagCA15/cat\_fullcutz\_fj1Tau32}
   \centering
   \includegraphics[width=0.7\linewidth]{170207_VTagCA15/cat_fullcutz_fj1Tau32.pdf}
\end{frame}

\begin{frame}
   \frametitle{\small 170207\_VTagCA15/cat\_fullcutz\_medB}
   \centering
   \includegraphics[width=0.7\linewidth]{170207_VTagCA15/cat_fullcutz_medB.pdf}
\end{frame}

\begin{frame}
   \frametitle{\small 170207\_VTagCA15/cat\_fullcutz\_looseB}
   \centering
   \includegraphics[width=0.7\linewidth]{170207_VTagCA15/cat_fullcutz_looseB.pdf}
\end{frame}

\begin{frame}
   \frametitle{\small 170207\_VTagCA15/cat\_fullcutz\_fj1Tau32SD}
   \centering
   \includegraphics[width=0.7\linewidth]{170207_VTagCA15/cat_fullcutz_fj1Tau32SD.pdf}
\end{frame}

\begin{frame}
   \frametitle{\small 170207\_VTagCA15/cat\_fullcutz\_N2DDT}
   \centering
   \includegraphics[width=0.7\linewidth]{170207_VTagCA15/cat_fullcutz_N2DDT.pdf}
\end{frame}

\begin{frame}
   \frametitle{\small 170207\_VTagCA15\_background/cat\_fullcutz\_N2DDT}
   \centering
   \includegraphics[width=0.7\linewidth]{170207_VTagCA15_background/cat_fullcutz_N2DDT.pdf}
\end{frame}

\begin{frame}
   \frametitle{\small 170207\_VTagCA15\_more/cat\_fullcutz\_N2DDT}
   \centering
   \includegraphics[width=0.7\linewidth]{170207_VTagCA15_more/cat_fullcutz_N2DDT.pdf}
\end{frame}

\begin{frame}
   \frametitle{\small 170207\_VTagCA15\_down/cat\_fullcutz\_N2DDT}
   \centering
   \includegraphics[width=0.7\linewidth]{170207_VTagCA15_down/cat_fullcutz_N2DDT.pdf}
\end{frame}

\begin{frame}
   \frametitle{\small 170207\_VTagCA15\_up/cat\_fullcutz\_N2DDT}
   \centering
   \includegraphics[width=0.7\linewidth]{170207_VTagCA15_up/cat_fullcutz_N2DDT.pdf}
\end{frame}

\begin{frame}
   \frametitle{\small 170207\_VTagCA15/cat\_fullcutz\_fj1MaxCSV}
   \centering
   \includegraphics[width=0.7\linewidth]{170207_VTagCA15/cat_fullcutz_fj1MaxCSV.pdf}
\end{frame}

\begin{frame}
   \frametitle{\small 170207\_VTagCA15/cat\_fullcutz\_dphiUW}
   \centering
   \includegraphics[width=0.7\linewidth]{170207_VTagCA15/cat_fullcutz_dphiUW.pdf}
\end{frame}

\begin{frame}
   \frametitle{\small 170207\_VTagCA15/cat\_base\_dRfj1Isob}
   \centering
   \includegraphics[width=0.7\linewidth]{170207_VTagCA15/cat_base_dRfj1Isob.pdf}
\end{frame}

\begin{frame}
   \frametitle{\small 170207\_VTagCA15/cat\_fullcutz\_dRfj1Isob}
   \centering
   \includegraphics[width=0.7\linewidth]{170207_VTagCA15/cat_fullcutz_dRfj1Isob.pdf}
\end{frame}

\begin{frame}
   \frametitle{\small 170207\_VTagCA15/cat\_base\_fj1MSD\_corr}
   \centering
   \includegraphics[width=0.7\linewidth]{170207_VTagCA15/cat_base_fj1MSD_corr.pdf}
\end{frame}

\begin{frame}
   \frametitle{\small 170207\_VTagCA15/cat\_fullcutz\_fj1MSD\_corr}
   \centering
   \includegraphics[width=0.7\linewidth]{170207_VTagCA15/cat_fullcutz_fj1MSD_corr.pdf}
\end{frame}

\begin{frame}
   \frametitle{\small 170207\_VTagCA15\_background/cat\_fullcutz\_fj1MSD\_corr}
   \centering
   \includegraphics[width=0.7\linewidth]{170207_VTagCA15_background/cat_fullcutz_fj1MSD_corr.pdf}
\end{frame}

\begin{frame}
   \frametitle{\small 170207\_VTagCA15\_more/cat\_fullcutz\_fj1MSD\_corr}
   \centering
   \includegraphics[width=0.7\linewidth]{170207_VTagCA15_more/cat_fullcutz_fj1MSD_corr.pdf}
\end{frame}

\begin{frame}
   \frametitle{\small 170207\_VTagCA15\_down/cat\_fullcutz\_fj1MSD\_corr}
   \centering
   \includegraphics[width=0.7\linewidth]{170207_VTagCA15_down/cat_fullcutz_fj1MSD_corr.pdf}
\end{frame}

\begin{frame}
   \frametitle{\small 170207\_VTagCA15\_up/cat\_fullcutz\_fj1MSD\_corr}
   \centering
   \includegraphics[width=0.7\linewidth]{170207_VTagCA15_up/cat_fullcutz_fj1MSD_corr.pdf}
\end{frame}

\begin{frame}
   \frametitle{\small 170207\_VTagCA15/cat\_fullcutz\_fj1Pt}
   \centering
   \includegraphics[width=0.7\linewidth]{170207_VTagCA15/cat_fullcutz_fj1Pt.pdf}
\end{frame}

\begin{frame}
   \frametitle{\small 170207\_VTagCA15/cat\_fullcutz\_nJet}
   \centering
   \includegraphics[width=0.7\linewidth]{170207_VTagCA15/cat_fullcutz_nJet.pdf}
\end{frame}

\begin{frame}
   \frametitle{\small 170207\_VTagCA15/cat\_fullcutz\_pfmetnomu}
   \centering
   \includegraphics[width=0.7\linewidth]{170207_VTagCA15/cat_fullcutz_pfmetnomu.pdf}
\end{frame}



\backupend

\end{document}
