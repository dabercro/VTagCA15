\documentclass{beamer}

\author[D. Abercrombie]{
  Daniel Abercrombie
}

\title{\bf \sffamily Rough V-tagging with CA15 Jets}
\date{30 January 2017}

\usecolortheme{dove}

\usepackage[absolute,overlay]{textpos}
\usefonttheme{serif}
\usepackage{appendixnumberbeamer}
\usepackage{isotope}
\usepackage{hyperref}
\usepackage[english]{babel}
\usepackage{amsmath}
\setbeamerfont{frametitle}{size=\Large,series=\bf\sffamily}
\setbeamertemplate{frametitle}[default][center]
\usepackage{siunitx}
\usepackage{tabularx}
\usepackage{makecell}

\setbeamertemplate{navigation symbols}{}
\usepackage{graphicx}
\usepackage{color}
\setbeamertemplate{footline}[text line]{\parbox{1.083\linewidth}{\footnotesize \hfill \insertshortauthor \hfill \insertpagenumber /\inserttotalframenumber}}
\setbeamertemplate{headline}[text line]{\parbox{1.083\linewidth}{\footnotesize \hspace{-0.083\linewidth} \textcolor{blue}{\sffamily \insertsection \hfill \insertsubsection}}}

\logo{\includegraphics[height=0.5cm]{/home/dabercro/MIT-logo.pdf}}

\usepackage{changepage}

\newcommand{\beginbackup}{
  \newcounter{framenumbervorappendix}
  \setcounter{framenumbervorappendix}{\value{framenumber}}
}
\newcommand{\backupend}{
  \addtocounter{framenumbervorappendix}{-\value{framenumber}}
  \addtocounter{framenumber}{\value{framenumbervorappendix}}
}

\newcommand{\tt}{\ensuremath{t\bar{t}}}

\graphicspath{{figs/}}

\begin{document}

\begin{frame}[nonumbering]
  \titlepage
\end{frame}

\begin{frame}
  \frametitle{Introduction}

  In a gross attempt to make my life simple, I explored a cut and count V-tag scale factor measurement
  in semileptonic \tt using CA15 jets.
  The peaks and backgrounds do not look as nice as the previous AK8 measurements we have seen,
  but the results are still promising.

\end{frame}

% Introduce the samples used. Deflect questions to Benedikt.
\begin{frame}
  \frametitle{Samples Used (Sort of)}

  Benedikt can hopefully answer questions about the details, but the main backgrounds used are:

  \begin{itemize}
  \item \tt + jets
  \item Single top
  \item W/Z + jets
  \item Diboson
  \item QCD
  \end{itemize}

\end{frame}

% Give the base selection
\begin{frame}
  \frametitle{Selection}

  Doesn't quite match the previous V-tagging

  \begin{itemize}
  \item One fat jet with $p_T > \SI{200}{GeV}$ and $50 < m_{SD} < \SI{250}{GeV}$
  \item One tight muon and no electrons, taus, or photons
  \item $\Delta\phi\mathrm{(fat jet, recoil)} > 0.4$
  \item Two isolated (ask Benedikt), loose-tagged b jets, with at least one medium
  \item Fewer than 4 isolated narrow (AK4) jets
  \item Passes MET filter
  \end{itemize}

\end{frame}

% Show a mass plot
\begin{frame}
  \frametitle{Soft Drop Mass}

  \begin{center}
  \includegraphics[width=0.6\linewidth]{170127/cat_base_fj1MSD_corr.pdf}
  \end{center}

  Yes, I know it doesn't look so good, now we're going to explore why that might be.

\end{frame}

% Show the DR fat, bjet to explain bad mass plot
\begin{frame}
  \frametitle{$\Delta R\mathrm{(fat jet, b jet)}$}

  \begin{columns}
    \begin{column}{0.5\linewidth}
      \centering
      \textcolor{blue}{Base selection}
      \includegraphics[width=\linewidth]{170127/cat_base_dRfj1Isob.pdf}
    \end{column}
    \begin{column}{0.5\linewidth}
      \centering
      \textcolor{blue}{Jets with the right mass}
      \includegraphics[width=\linewidth]{170127/cat_base+rightMass_dRfj1Isob.pdf}
    \end{column}
  \end{columns}

  The cone size is so large that b jets are not really falling just outside the cone
  like they were for the AK8 case.
  We can't do the same cuts as before.

\end{frame}

% Show how SD seems to cut out some of the b jets
\begin{frame}
  \frametitle{B jet removal}

  One thing we can look into is removing PF candidates from a b-tagged jet.
  This would also allow for us to measure more boosted jets in the AK8 case.

  \textcolor{red}{The following have $65 < m_{SD} < \SI{105}{GeV}$}

  \begin{columns}
    \begin{column}{0.5\linewidth}
      \centering
      \textcolor{blue}{Regular N-subjettiness}
      \includegraphics[width=\linewidth]{170127/cat_base+rightMass_fj1Tau21.pdf}
    \end{column}
    \begin{column}{0.5\linewidth}
      \centering
      \textcolor{blue}{With Soft Drop removal}
      \includegraphics[width=\linewidth]{170127/cat_base+rightMass_fj1Tau21SD.pdf}
    \end{column}
  \end{columns}

  Here, you can see at the correct Soft Drop mass, the algorithm probably removes particles from the b-jet, allowing for more conventional tagging.

\end{frame}

% Do scale factor measurement with systematics up and down

\beginbackup

\begin{frame}
  \frametitle{Backup Slides}
\end{frame}

\begin{frame}
   \frametitle{\small 170207\_VTagCA15/cat\_fullcutz\_fj1Tau32}
   \centering
   \includegraphics[width=0.7\linewidth]{170207_VTagCA15/cat_fullcutz_fj1Tau32.pdf}
\end{frame}

\begin{frame}
   \frametitle{\small 170207\_VTagCA15/cat\_fullcutz\_medB}
   \centering
   \includegraphics[width=0.7\linewidth]{170207_VTagCA15/cat_fullcutz_medB.pdf}
\end{frame}

\begin{frame}
   \frametitle{\small 170207\_VTagCA15/cat\_fullcutz\_looseB}
   \centering
   \includegraphics[width=0.7\linewidth]{170207_VTagCA15/cat_fullcutz_looseB.pdf}
\end{frame}

\begin{frame}
   \frametitle{\small 170207\_VTagCA15/cat\_fullcutz\_fj1Tau32SD}
   \centering
   \includegraphics[width=0.7\linewidth]{170207_VTagCA15/cat_fullcutz_fj1Tau32SD.pdf}
\end{frame}

\begin{frame}
   \frametitle{\small 170207\_VTagCA15/cat\_fullcutz\_N2DDT}
   \centering
   \includegraphics[width=0.7\linewidth]{170207_VTagCA15/cat_fullcutz_N2DDT.pdf}
\end{frame}

\begin{frame}
   \frametitle{\small 170207\_VTagCA15\_background/cat\_fullcutz\_N2DDT}
   \centering
   \includegraphics[width=0.7\linewidth]{170207_VTagCA15_background/cat_fullcutz_N2DDT.pdf}
\end{frame}

\begin{frame}
   \frametitle{\small 170207\_VTagCA15\_more/cat\_fullcutz\_N2DDT}
   \centering
   \includegraphics[width=0.7\linewidth]{170207_VTagCA15_more/cat_fullcutz_N2DDT.pdf}
\end{frame}

\begin{frame}
   \frametitle{\small 170207\_VTagCA15\_down/cat\_fullcutz\_N2DDT}
   \centering
   \includegraphics[width=0.7\linewidth]{170207_VTagCA15_down/cat_fullcutz_N2DDT.pdf}
\end{frame}

\begin{frame}
   \frametitle{\small 170207\_VTagCA15\_up/cat\_fullcutz\_N2DDT}
   \centering
   \includegraphics[width=0.7\linewidth]{170207_VTagCA15_up/cat_fullcutz_N2DDT.pdf}
\end{frame}

\begin{frame}
   \frametitle{\small 170207\_VTagCA15/cat\_fullcutz\_fj1MaxCSV}
   \centering
   \includegraphics[width=0.7\linewidth]{170207_VTagCA15/cat_fullcutz_fj1MaxCSV.pdf}
\end{frame}

\begin{frame}
   \frametitle{\small 170207\_VTagCA15/cat\_fullcutz\_dphiUW}
   \centering
   \includegraphics[width=0.7\linewidth]{170207_VTagCA15/cat_fullcutz_dphiUW.pdf}
\end{frame}

\begin{frame}
   \frametitle{\small 170207\_VTagCA15/cat\_base\_dRfj1Isob}
   \centering
   \includegraphics[width=0.7\linewidth]{170207_VTagCA15/cat_base_dRfj1Isob.pdf}
\end{frame}

\begin{frame}
   \frametitle{\small 170207\_VTagCA15/cat\_fullcutz\_dRfj1Isob}
   \centering
   \includegraphics[width=0.7\linewidth]{170207_VTagCA15/cat_fullcutz_dRfj1Isob.pdf}
\end{frame}

\begin{frame}
   \frametitle{\small 170207\_VTagCA15/cat\_base\_fj1MSD\_corr}
   \centering
   \includegraphics[width=0.7\linewidth]{170207_VTagCA15/cat_base_fj1MSD_corr.pdf}
\end{frame}

\begin{frame}
   \frametitle{\small 170207\_VTagCA15/cat\_fullcutz\_fj1MSD\_corr}
   \centering
   \includegraphics[width=0.7\linewidth]{170207_VTagCA15/cat_fullcutz_fj1MSD_corr.pdf}
\end{frame}

\begin{frame}
   \frametitle{\small 170207\_VTagCA15\_background/cat\_fullcutz\_fj1MSD\_corr}
   \centering
   \includegraphics[width=0.7\linewidth]{170207_VTagCA15_background/cat_fullcutz_fj1MSD_corr.pdf}
\end{frame}

\begin{frame}
   \frametitle{\small 170207\_VTagCA15\_more/cat\_fullcutz\_fj1MSD\_corr}
   \centering
   \includegraphics[width=0.7\linewidth]{170207_VTagCA15_more/cat_fullcutz_fj1MSD_corr.pdf}
\end{frame}

\begin{frame}
   \frametitle{\small 170207\_VTagCA15\_down/cat\_fullcutz\_fj1MSD\_corr}
   \centering
   \includegraphics[width=0.7\linewidth]{170207_VTagCA15_down/cat_fullcutz_fj1MSD_corr.pdf}
\end{frame}

\begin{frame}
   \frametitle{\small 170207\_VTagCA15\_up/cat\_fullcutz\_fj1MSD\_corr}
   \centering
   \includegraphics[width=0.7\linewidth]{170207_VTagCA15_up/cat_fullcutz_fj1MSD_corr.pdf}
\end{frame}

\begin{frame}
   \frametitle{\small 170207\_VTagCA15/cat\_fullcutz\_fj1Pt}
   \centering
   \includegraphics[width=0.7\linewidth]{170207_VTagCA15/cat_fullcutz_fj1Pt.pdf}
\end{frame}

\begin{frame}
   \frametitle{\small 170207\_VTagCA15/cat\_fullcutz\_nJet}
   \centering
   \includegraphics[width=0.7\linewidth]{170207_VTagCA15/cat_fullcutz_nJet.pdf}
\end{frame}

\begin{frame}
   \frametitle{\small 170207\_VTagCA15/cat\_fullcutz\_pfmetnomu}
   \centering
   \includegraphics[width=0.7\linewidth]{170207_VTagCA15/cat_fullcutz_pfmetnomu.pdf}
\end{frame}



\backupend

\end{document}
